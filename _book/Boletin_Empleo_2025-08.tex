% Options for packages loaded elsewhere
\PassOptionsToPackage{unicode}{hyperref}
\PassOptionsToPackage{hyphens}{url}
%
\documentclass[
]{book}
\usepackage{amsmath,amssymb}
\usepackage{iftex}
\ifPDFTeX
  \usepackage[T1]{fontenc}
  \usepackage[utf8]{inputenc}
  \usepackage{textcomp} % provide euro and other symbols
\else % if luatex or xetex
  \usepackage{unicode-math} % this also loads fontspec
  \defaultfontfeatures{Scale=MatchLowercase}
  \defaultfontfeatures[\rmfamily]{Ligatures=TeX,Scale=1}
\fi
\usepackage{lmodern}
\ifPDFTeX\else
  % xetex/luatex font selection
\fi
% Use upquote if available, for straight quotes in verbatim environments
\IfFileExists{upquote.sty}{\usepackage{upquote}}{}
\IfFileExists{microtype.sty}{% use microtype if available
  \usepackage[]{microtype}
  \UseMicrotypeSet[protrusion]{basicmath} % disable protrusion for tt fonts
}{}
\makeatletter
\@ifundefined{KOMAClassName}{% if non-KOMA class
  \IfFileExists{parskip.sty}{%
    \usepackage{parskip}
  }{% else
    \setlength{\parindent}{0pt}
    \setlength{\parskip}{6pt plus 2pt minus 1pt}}
}{% if KOMA class
  \KOMAoptions{parskip=half}}
\makeatother
\usepackage{xcolor}
\usepackage{longtable,booktabs,array}
\usepackage{calc} % for calculating minipage widths
% Correct order of tables after \paragraph or \subparagraph
\usepackage{etoolbox}
\makeatletter
\patchcmd\longtable{\par}{\if@noskipsec\mbox{}\fi\par}{}{}
\makeatother
% Allow footnotes in longtable head/foot
\IfFileExists{footnotehyper.sty}{\usepackage{footnotehyper}}{\usepackage{footnote}}
\makesavenoteenv{longtable}
\usepackage{graphicx}
\makeatletter
\newsavebox\pandoc@box
\newcommand*\pandocbounded[1]{% scales image to fit in text height/width
  \sbox\pandoc@box{#1}%
  \Gscale@div\@tempa{\textheight}{\dimexpr\ht\pandoc@box+\dp\pandoc@box\relax}%
  \Gscale@div\@tempb{\linewidth}{\wd\pandoc@box}%
  \ifdim\@tempb\p@<\@tempa\p@\let\@tempa\@tempb\fi% select the smaller of both
  \ifdim\@tempa\p@<\p@\scalebox{\@tempa}{\usebox\pandoc@box}%
  \else\usebox{\pandoc@box}%
  \fi%
}
% Set default figure placement to htbp
\def\fps@figure{htbp}
\makeatother
\setlength{\emergencystretch}{3em} % prevent overfull lines
\providecommand{\tightlist}{%
  \setlength{\itemsep}{0pt}\setlength{\parskip}{0pt}}
\setcounter{secnumdepth}{5}
\usepackage{fancyhdr}
\pagestyle{fancy}

% Captura el título del capítulo
\renewcommand{\chaptermark}[1]{\markboth{#1}{}}

% Encabezado dinámico
\fancyhead[L]{\leftmark}
\fancyhead[C]{}
\fancyhead[R]{\today}

% Pie de página
\fancyfoot[L]{Confidencial}
\fancyfoot[C]{}
\fancyfoot[R]{Página \thepage}

% Líneas decorativas
\renewcommand{\headrulewidth}{0.4pt}
\renewcommand{\footrulewidth}{0.4pt}

\usepackage{booktabs}
\usepackage{longtable}
\usepackage{array}
\usepackage{multirow}
\usepackage{wrapfig}
\usepackage{float}
\usepackage{colortbl}
\usepackage{pdflscape}
\usepackage{tabu}
\usepackage{threeparttable}
\usepackage{threeparttablex}
\usepackage[normalem]{ulem}
\usepackage{makecell}
\usepackage{xcolor}
\usepackage{bookmark}
\IfFileExists{xurl.sty}{\usepackage{xurl}}{} % add URL line breaks if available
\urlstyle{same}
\hypersetup{
  pdftitle={Untitled},
  pdfauthor={Diego},
  hidelinks,
  pdfcreator={LaTeX via pandoc}}

\title{Untitled}
\author{Diego}
\date{2025-08-18}

\begin{document}
\maketitle

{
\setcounter{tocdepth}{1}
\tableofcontents
}
\thispagestyle{empty}

\chapter{Portada}\label{portada}

Aquí va el resumen institucional, logotipo, etc.

\newpage

\thispagestyle{empty}

\section{R Markdown}\label{r-markdown}

This is an R Markdown document. Markdown is a simple formatting syntax for authoring HTML, PDF, and MS Word documents. For more details on using R Markdown see \url{http://rmarkdown.rstudio.com}.

When you click the \textbf{Knit} button a document will be generated that includes both content as well as the output of any embedded R code chunks within the document. You can embed an R code chunk like this:

\begin{verbatim}
##      speed           dist       
##  Min.   : 4.0   Min.   :  2.00  
##  1st Qu.:12.0   1st Qu.: 26.00  
##  Median :15.0   Median : 36.00  
##  Mean   :15.4   Mean   : 42.98  
##  3rd Qu.:19.0   3rd Qu.: 56.00  
##  Max.   :25.0   Max.   :120.00
\end{verbatim}

\section{Including Plots}\label{including-plots}

You can also embed plots, for example:

\pandocbounded{\includegraphics[keepaspectratio]{Boletin_Empleo_2025-08_files/figure-latex/pressure-1.pdf}}

Note that the \texttt{echo\ =\ FALSE} parameter was added to the code chunk to prevent printing of the R code that generated the plot.

\chapter{Servicios de Capacitación y Formación}\label{servicios-de-capacitaciuxf3n-y-formaciuxf3n}

Procesos de Capacitación

\begin{table}[!h]
\centering
\caption{\label{tab:capacitacion}Personas capacitadas por trimestre}
\centering
\begin{tabular}[t]{l|r|r}
\hline
Trimestre & Capacitados & Porcentaje\\
\hline
I & 684 & 12\\
\hline
II & 1980 & 34\\
\hline
III & 2026 & 35\\
\hline
IV & 1070 & 19\\
\hline
\end{tabular}
\end{table}

\section{Campos de Conocimiento}\label{campos-de-conocimiento}

\pandocbounded{\includegraphics[keepaspectratio]{Boletin_Empleo_2025-08_files/figure-latex/campos-conocimiento-1.pdf}}

\chapter{Perfil de los Participantes}\label{perfil-de-los-participantes}

\begin{table}[!h]
\centering
\caption{\label{tab:perifl-participantes}Distribución de usuarios por características demográficas}
\centering
\begin{tabular}[t]{l|r}
\hline
Variable & Porcentaje\\
\hline
Ecuatoriano/a & 95.17\\
\hline
Extranjero/a & 2.52\\
\hline
No registra & 2.31\\
\hline
18-29 años & 34.48\\
\hline
30-40 años & 31.15\\
\hline
41-55 años & 25.57\\
\hline
56+ & 8.13\\
\hline
<18 & 0.59\\
\hline
Femenino & 55.36\\
\hline
Masculino & 44.50\\
\hline
\end{tabular}
\end{table}

\chapter{Ferias de Empleo}\label{ferias-de-empleo}

\begin{table}[!h]
\centering
\caption{\label{tab:ferias-empleo}Resumen de ferias de empleo 2024}
\centering
\begin{tabular}[t]{l|r|r|r|r}
\hline
Mes & Asistentes & Contratados & Vacantes & Empresas\\
\hline
Agosto & 12131 & 337 & 2273 & 65\\
\hline
Octubre & 6385 & 220 & 1223 & 50\\
\hline
\end{tabular}
\end{table}

\chapter{Bolsa Metropolitana de Empleo}\label{bolsa-metropolitana-de-empleo}

\begin{table}[!h]
\centering
\caption{\label{tab:bme}Indicadores de la Bolsa Metropolitana de Empleo}
\centering
\begin{tabular}[t]{l|r}
\hline
Indicador & Total\\
\hline
Candidatos registrados & 18492\\
\hline
Nuevos candidatos & 2026\\
\hline
Empresas registradas & 229\\
\hline
Nuevas empresas & 27\\
\hline
\end{tabular}
\end{table}

\chapter{Escuela Taller Quito II}\label{escuela-taller-quito-ii}

\pandocbounded{\includegraphics[keepaspectratio]{Boletin_Empleo_2025-08_files/figure-latex/escuela-taller-1.pdf}}

\chapter{Proyectos Especiales}\label{proyectos-especiales}

Emprendimiento: 100 beneficiarios, 33\% mujeres, 67\% hombres.

Escolaridad: 83 niños/as beneficiados, 69 familias.

Capacitación vocacional: 106 personas, incluyendo 10 extranjeras.

\chapter{Satisfacción de Usuarios}\label{satisfacciuxf3n-de-usuarios}

\pandocbounded{\includegraphics[keepaspectratio]{Boletin_Empleo_2025-08_files/figure-latex/satisfaccion-usuarios-1.pdf}}

\end{document}
